\documentclass[12pt]{article}

\begin{document}
\title{Activity V: Planets}
\maketitle

\section{Introduction and Objective}

In this activity you will answer the following questions regarding the planets in our Solar System. You may refer to the lecture videos, e-book, or, if you like, to any valid web resources. quasi-open-ended

\begin{enumerate}
    \item List all of the planets in our Solar System in order of their proximity to the Sun.
        \begin{enumerate}
            \item Mercury
            \item Venus
            \item Earth
            \item Mars
            \item Jupiter
            \item Saturn
            \item Uranus
            \item Neptune
        \end{enumerate}
    \item List all of the planets in our Solar System in order of their average surface temperature.\newline\newline
        (Hottest to coldest)
        \begin{enumerate}
            \item Venus
            \item Mercury
            \item Earth
            \item Mars
            \item Jupiter
            \item Saturn
            \item Uranus
            \item Neptune
        \end{enumerate}
    \item List all of the planets in our Solar System in order of their radii.
        \newline\newline
        (Largest to smallest)
        \begin{enumerate}
            \item Jupiter
            \item Saturn
            \item Uranus
            \item Neptune
            \item Earth
            \item Venus
            \item Mars
            \item Mercury
        \end{enumerate}
    \item Jupiter’s moon Europa very likely has a liquid water ocean beneath its icy surface. Why doesn’t this liquid ocean freeze?
        \newline\newline
        Because of the hydrothermal vents underneath the subsurface ocean.
    \item List the rocky planets in descending order of their surface pressures (i.e., from highest pressure to lowest).
        \begin{enumerate}
            \item Venus
            \item Earth
            \item Mars
            \item Mercury
        \end{enumerate}
    \item List the rocky planets in descending order of their masses.
        \begin{enumerate}
            \item Earth %$1.317 \times 10^{25}$ lbs
            \item Venus  %$1.073 \times 10^{25}$ lbs
            \item Mars %$1.415 \times 10^{25}$ lbs
            \item Mercury  %$7.278 \times 10^{23}$ lbs
        \end{enumerate}
    \item Define the terms planet and dwarf planet.\newline\newline
        Planet: A celestial body that orbits a star and has enough mass to have its own gravity. Planets are neither stars or satellites of other planets.\newline\newline
        Dwarf Planet: A dwarf planet is like a planet with sufficient mass but it's size in diameter must be small compared to other planets' size.
    \item The Moon’s mass is 0.0123 of Earth’s, and its radius is 0.2727 of Earth’s. Using this information, determine the surface gravity on the Moon compared to that on Earth.\newline\newline
        Moons gravitational acceleration:
        \begin{equation}
          a_{g} = \frac{GM}{r^{2}}
        \end{equation}
        \begin{equation}
          a_{g} = \frac{(6.67408\times 10^{-11}) (7.345806\times10^{22})}{(1739.2806)^{2}}
        \end{equation}
        \begin{equation}
          a_{g} = \frac{(4.897114\times10^{12})}{(1739)^{2}}
        \end{equation}
        \begin{equation}
          a_{g} = 1.623 \times 10^{6}
        \end{equation}

        Therefore, moons gravity is  1.62$ m/s^{2}$. Earths gravity is 9.81 $m/s^{2}$. Which means moons gravity is 0.165 that of earth.



    \item State three discoveries that Curiosity has made.
        \begin{enumerate}
          \item Found large quantities of water on mars.
          \item Discovered organic molecules in martian rock.
          \item Detected radiation levels to be too dangerous for humans.
        \end{enumerate}
    \item If the Moon were 1 mile from Earth, how far from the Earth would Mars be?\newline\newline
        1300 miles.
\end{enumerate}

$fill$ $in$ $the$ $blank$

\begin{enumerate}
    \item The main constituent of the Martian atmosphere is \underline{Carbon Dioxide}
    \item The main constituent of Jupiter is \underline{Hydrogen}
    \item The Venusian atmosphere \underline{92} times as thick as Earth’s atmosphere (by pressure).
    \item With our current technology, it takes about \underline{6} months to get to Mars.
    \item There \underline{is} evidence that Mars currently has an abundance of liquid water on its surface.
    \item The four largest moons of Jupiter are \underline{Ganymede, Callisto, Io, and Europa}
    \item The Venusian day lasts for \underline{5832} hours.
    \item The planet with the largest day-night fluctuation in temperature is \underline{Mercury}.
\end{enumerate}

————————————————
\end{document}
