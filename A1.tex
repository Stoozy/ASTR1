\documentclass[12pt]{article}



\begin{document}

\title{Activity I: The Celestial Sphere}

\maketitle


\section {Introduction and Objective}

In this activity you will answer the following questions regarding the celestial
sphere. All of this material was discussed in the lecture videos.\newline

\flushleft{$essential$ $concepts$\newline}

For each statement, fill in the blank with the most appropriate term(s).\newline

\begin{enumerate}
        \item The name of the region on the celestial sphere where all of the stars are visible to a stargazer (in the northern hemisphere) all year round is called \underline{zenith}.

        \item The number of times the path of the Sun on the celestial sphere crosses
the celestial equator is \underline{2 times}.

        \item The number of days during the year when there are equal amounts (in
terms of time) of light and dark is \underline{2 days}.

        \item The sidereal day is different than the solar day because of \underline{the rotation} of the Earth.

        \item The Sun moves \underline{about 1} degrees around the ecliptic per day.

        \item The North Celestial Pole is at the zenith only at a latitude \underline{90 degrees} of on Earth.

\flushleft{$open-ended$ $questions$}

        Suppose you are stargazing at a latitude of 30.0◦ on Earth. For each of
the following stars, determine if it is (a) visible all year round, (b) visible
sometimes during the year, or (c) never visible. For each, explain your reasoning.
\begin{enumerate}
        \item (star i at a declination of 70◦)\newline\newline
        For it to be visible all year round, the following has to be true:\newline
        \centerline{$\theta + \delta  > 90\circ$}
        where $\theta$ = observers lattitude and $\delta$ = declination of the star
        \centerline{$30\circ + 70\circ > 90\circ$}
        \centerline{$ 100\circ > 90\circ$}
        Therfore, star i is visible all year round\newline

        \item (star ii, at a declination of 30◦ )\newline
        Same goes for this one.
        \centerline{$\theta + \delta > 90\circ$}
        \centerline{$30 + 30 > 90\circ$}
        This is not true which means, it's not always visible.\newline
        Now let's see if it's never visible.
        \centerline{$\theta - \delta < -90\circ$}
        \centerline{$30 - 30 < -90\circ$}
        \centerline{$0 < -90\circ$}
        This is also not true, which means the star must be visible sometimes throughout the year.\newline

        \item (star iii, at a declination of $-30$◦ )
        \centerline{$\theta + \delta > 90\circ$}
        \centerline{$30 + (-30) > 90\circ$}
        \centerline{$ 0 > 90\circ$}
        This means, the star is not always visible.
        Now lets see if it's never visible,
        \centerline{$\theta - \delta < -90\circ$}
        \centerline{$30 - (-30) < -90\circ$}
        \centerline{$30 + 30 < -90\circ$}
        \centerline{$60 < -90\circ$}
        This is false and therefore, star iii is sometimes visible.

        \item (star iv, at a declination of $-80$◦ )
        \centerline{$\theta + \delta > 90\circ$}
        \centerline{$ 30 + (-80) > 90\circ$}
        \centerline{$ -50 > 90\circ$}
        This means the star is not always visible.
        Now let's check if it's never visible,
        \centerline{$\theta - \delta < -90\circ$}
        \centerline{$30 - (-80) < -90\circ$}
        \centerline{$110 < -90\circ$}
        This is false and therefore, star iv is sometimes visible.


\end{enumerate}

2. Will the pole star that we see in the northern hemisphere always remain the same? If not, why not? Explain.\newline\newline
        I don't think so because the earth is at a tilt. The tilt also rotates which means the pole star may not always be at the same location.

3. Suppose that you are viewing star A at 11:32 PM on September 2. You note where the star is in the sky at this time. Your friend visits you on September 13. At what time should you both look for this star in the same spot in the sky? Explain.\newline\newline
        Each sidereal day is around 23 hours and 56 minutes, which means the time will be 4 minutes less the next day. For 11 days, it would be 11*4 or 44 minutes less. Which is 10:48 PM on the 13th.


\end{enumerate}
————————————————


\end{document}
